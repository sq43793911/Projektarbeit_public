	\pagestyle{fancy}
	\fancyhead[RO,LE]{Anhang A: Programm Stabelemente}
	\setcounter{page}{1} 
	\fancyhead[LO,RE]{A-\thepage}
	%\specialsectioning
	\addcontentsline{toc}{paragraph}{Anhang A: Programm Stabelemente}
	\section*{Anhang A: Programm Stabelemente}
	\subsection*{Hauptprogramm linearer Ansatz}
	\begin{lstlisting}
	clear
	clc
	close
	%###########################################
	% main Program
	%###########################################
	% parameter
	E=2.1e11;           % N/m^2
	A=0.0001;           % m^2
	l=10;               % m
	rho=7850;           % Dichte in [kg/m^3]
	mu=rho*A;           % Massenbelegung in [kg/m]
	Nel=15;             % number of elements
	Nno=Nel+1;          % number of nodes
	le=l/Nel;           % length of an element
	
	% define empty matrice
	Kt=zeros(Nno);   % empty global stiffnes-matrix 
	M=zeros(Nno);    % empty global mass-matrix 
	
	% call element routinesmbclient
	[Kte,Me] = Elementroutine_linear(A,E,rho,le);
	
	% loop over every element
	for j=1:Nel  
	  M(j : j+1, j : j+1) = M(j : j+1, j : j+1) + Me;
	  Kt(j : j+1, j : j+1) = Kt(j : j+1, j : j+1)+ Kte;	
	end
	
	% implementation of essetial boundary conditions
	Kt(1,:) = [  ];
	Kt(:,1) = [  ];
	M(1,:)  = [  ];
	M(:,1)  = [  ];
	
	% define system-matrix
	null=zeros(size(M));
	Eins = eye(size(M));
	SysMat=[null,Eins; -inv(M)*Kt,null];
	
	% compute Eigenvalues
	[V,D]=eig(SysMat);
	temp_d = diag(D);
	[nd, sortindex] = sort(temp_d);
	temp_v = V(:,sortindex);
	temp_f = imag(nd)/(2*pi);
	
	% analysis method
	lamda = zeros(Nel, 1);
	omega = zeros(Nel, 1);
	f = zeros (Nel, 1);
	for k = 1: Nel
	 lamda (k) = (2 * k - 1 ) * pi /( 2 * l) ;
	 omega (k) = lamda(k)*sqrt(E/rho);
	 f(k) = omega(k)/(2*pi);
	end
	\end{lstlisting}

	\subsubsection*{Elementrechnung linearer Ansatz}
	\begin{lstlisting}
	%###########################################
	% Elementroutine
	%###########################################
	function [Kte,Me] = Elementroutine_linear(A,E,rho,le)
	% Elementroutine: compute Kte, Me
	% define empty Kte
	Kte=[0,0;0,0]; 
	% define empty Me 
	Me=[0,0;0,0];
	
	% define sampling points for Gauss-quadrature
	xiVec=[-sqrt(1/3),sqrt(1/3)];
	% weights for sampling points of Gauss-quadrature
	wVec =[1,1];  
	
	for i=1:length(xiVec)
	 xi=xiVec(i);
	  w=wVec(i);
	
	 % define N, B vector 
	 N=[0.5-xi/2  0.5+xi/2 ];
	 Nx=[-0.5  0.5]*(2/le);
	
	 % compute Kte and Me for sampling point of Gauss-integration
	 Me=Me + rho * A * ( N' * N ) * w;
	 Kte=Kte + E * A * ( Nx' * Nx ) * w;
	end
	
	end
	\end{lstlisting}
	
	\subsection*{Hauptprogramm quadratischer Ansatz}
	\begin{lstlisting}
	clear
	close
	clc
	%###########################################
	% main Program
	%###########################################
	% parameter
	E=2.1e11;           % N/m^2
	A=0.0001;           % m^2
	l=10;               % m
	rho=7850;           % Dichte in [kg/m^3]
	mu=rho*A;           % Massenbelegung in [kg/m]
	Nel=15;             % number of elements
	Nno=Nel*2+1;        % number of nodes
	le=l/Nel;           % length of an element
	
	% define empty matrice
	Kt=zeros(Nno);   % empty global stiffnes-matrix 
	M=zeros(Nno);    % empty global mass-matrix 
	
	% call element routinesmbclient
	[Kte,Me] = Elementroutine_quadra(A,E,mu,le);
	% loop over every element
	for j=1:2 : Nno-1                                    
	 M(j : j+2, j : j+2) = M(j : j+2, j : j+2) + Me;
	 Kt(j : j+2, j : j+2) = Kt(j : j+2, j : j+2)+ Kte;   
	end
	
	% implementation of essetial boundary conditions
	Kt(1,:) = [  ];
	Kt(:,1) = [  ];
	M(1,:)  = [  ];
	M(:,1)  = [  ];
	
	% define system-matrix
	null=zeros(size(M));
	Eins = eye(size(M));
	SysMat=[null,Eins;-inv(M)*Kt,null];
	
	% compute Eigenvalues
	[V,D]=eig(SysMat);
	temp_d = diag(D);
	[nd, sortindex] = sort(temp_d);
	temp_v = V(:,sortindex);
	temp_f = imag(nd)/(2*pi);
	
	% analysis method
	lamda = zeros(18, 1);
	omega = zeros(18, 1);
	f = zeros (18, 1);
	for k = 1: 18
	 lamda (k) = (2 * k - 1 ) * pi /( 2 * l) ;
	 omega (k) = lamda(k)*sqrt(E/rho);
	 f(k) = omega(k)/(2*pi);
	end
	\end{lstlisting}
	
	\subsubsection*{Elementrechnung quadratischer Ansatz}
	\begin{lstlisting}
	%###########################################
	% Elementroutine
	%###########################################
	function [Kte,Me] = Elementroutine_quadra(A,E,mu,le)
	% Elementroutine: compute Kte, Me
	% define empty Kte
	Kte=zeros(3); 
	% define empty Me 
	Me=zeros(3);
	
	% define sampling points for Gauss-quadrature
	xiVec=[-sqrt(1/3),sqrt(1/3)]; 
	% weights for sampling points of Gauss-quadrature 
	wVec =[1,1];   
	
	for i=1:length(xiVec)
	 xi=xiVec(i);
	 w=wVec(i);
	
	 % define N, B vector 
	 N=[xi^2/2-xi/2  1-xi^2  xi/2+xi^2/2];
	 Nx=[xi-0.5  -2*xi  0.5+xi]*(2/le);
	
	 % compute Kte and Me for sampling point of Gauss-integration
	 Me=Me + mu * (N' * N) * w;
	 Kte=Kte + E * A * (Nx' * Nx) * w;
	end
	
	end
	\end{lstlisting}
	
	\clearpage
	
	\setcounter{page}{1} 
	\fancyhead[LO,RE]{B-\thepage}
	\fancyhead[RO,LE]{Anhang B: Programm Balkenelemente}
	\addcontentsline{toc}{subparagraph}{Anhang B: Programm Balkenelemente}
	
	\section*{Anhang B: Programm Balkenelemente}
	\subsection*{Hauptprogramm kubischer Ansatz}
	\begin{lstlisting}
	clear
	clc
	close
	%###########################################
	% main Program
	%###########################################
	% parameter
	E=2.1e11;        % N/m^2
	A=0.0001;        % m^2
	l=10;            % m
	rho=7850;        % Dichte in [kg/m^3]
	mu=rho*A;        % Massenbelegung in [kg/m]
	Nel=15;         % number of elements
	Nno=Nel+1;       % number of nodes
	le=l/Nel;        % length of an element
	B=0.005;         % Breit von Balken
	H=0.02;          % Hoehe von Balken
	I=(B*(H^3))/12;  % Flaechentraegheitsmoment
	
	% define empty matrice
	Kt=zeros(Nno*2);  % empty global stiffnes-matrix 
	M=zeros(Nno*2);   % empty global mass-matrix 
	
	% call element routinesmbclient
	[Kte,Me] = Elementroutine_Balken(A,E,rho,le,I);
	
	% loop over every element
	for j=1: 2 : length(M)-2                                   
	 M(j : j+3, j : j+3) = M(j : j+3, j : j+3) + Me;
	 Kt(j : j+3, j : j+3) = Kt(j : j+3, j : j+3)+ Kte;
	end
	
	% implementation of essetial boundary conditions
	Kt(1,:) = [  ];
	Kt(:,1) = [  ];
	M(1,:)  = [  ];
	M(:,1)  = [  ];
	
	Kt(1,:) = [  ];
	Kt(:,1) = [  ];
	M(1,:)  = [  ];
	M(:,1)  = [  ];
	
	% define system-matrix
	null=zeros(size(M));
	Eins = eye(size(M));
	SysMat=[null,Eins; -inv(M)*Kt,null];
	
	% compute Eigenvalues
	[V,D]=eig(SysMat);
	
	temp_d = diag(D);
	[nd, sortindex] = sort(temp_d);
	temp_v = V(:,sortindex);
	temp_f = imag(nd)/(2*pi);
	
	for i=1:Nel*2
	 temp_f(i,:)=[];
	end
	% vector with node coordinates
	lVec = zeros (Nno, 1); 	
	for k = 1: Nno
	 lVec(k)=l/Nel*(k-1);
	end
	% delete lover part von eigenvectors
	EigMat=temp_v(1:length(M),:) ; 
	EigMat=imag(EigMat);
	% delete unnecessary rows 
	for k = 1: length(M)/2      
	 EigMat(1+k,:)=[];       
	end
	% delete unnecessary colums
	for k = 1: length(M)
	EigMat(:,k)=[];        
	end
	% fill up EigMat for first with zero displacement
	EigMat=[zeros(1,length(M));EigMat]; 
	
	figure
	hold on
	grid on
	plot(lVec,-EigMat(:,1),'LineWidth',4);
	plot(lVec,EigMat(:,2),'--','LineWidth',4);
	plot(lVec,EigMat(:,3),':','LineWidth',4);
	set(gca,'FontSize',24);
	legend('1-Mode', '2-Mode', '3-Mode','Location','northwest');
	
	% analysis method
	lamda = zeros(Nel, 1);
	omega = zeros(Nel, 1);
	f = zeros (Nel, 1);
	for k = 1: Nel
	 switch k
	  case 1 
	   lamda(k) = 1.87510 ;
	  case 2
	   lamda(k) = 4.69409 ;
	  case 3
	   lamda(k) = 7.85476 ;
	  case 4
	   lamda(k) = 10.99554;
	 end
	 if k >= 5
	 lamda(k) = ((2*k-1)*pi)/2 ;
	 end	
	 omega (k) = ((lamda(k))^2)*sqrt((E*I)/(rho*A*(l^4)));
	 f(k)=omega(k)/(2*pi);
	end
	\end{lstlisting}
	
	\subsubsection*{Elementrechnung kubischer Ansatz}
	\begin{lstlisting}
	%###########################################
	% Elementroutine
	%###########################################
	function [Kte,Me] = Elementroutine_Balken(A,E,rho,le,I)
	% Elementroutine: compute Kte, Me
	% define empty Kte
	Kte=zeros(4);
	% define empty Me 
	Me=zeros(4);
	% define sampling points for Gauss-quadrature
	xiVec=[-sqrt(3/5),0,sqrt(3/5)]; 
	% weights for sampling points of Gauss-quadrature
	wVec=[5/9,8/9,5/9];            
	for i=1:length(xiVec)
	 xi=xiVec(i);
	 w =wVec(i);
	 % zwei Freiheitsgrade
	 N = [1/2-(3*xi)/4+(xi^3)/4   1/4-xi/4-(xi^2)/4+(xi^3)/4   1/2+(3*xi)/4-(xi^3)/4   -1/4-xi/4+(xi^2)/4+(xi^3)/4];
	 Nxx = [(3*xi)/2   -1/2+(3*xi)/2   -(3*xi)/2   1/2+(3*xi)/2]*((2/le)^2);
	 % compute Kte and Me for sampling point of Gauss-integration
	 Me=Me + rho * A * (N' * N) * w;
	 Kte=Kte + E * I * (Nxx' * Nxx) * w;
	end
	
	end
	\end{lstlisting}
	
	\clearpage
	
	% empty page %%%%%%%%%%%%%%%%%%%%%%%%%%%%%%%%%%%%%%%%%%
	\newpage
	\pagestyle{empty}
	\ \\
	\newpage
	%%%%%%%%%%%%%%%%%%%%%%%%%%%%%%%%%%%%%%%%%%%%%%%%%%%%%%%%
	
	\setcounter{page}{1} 
	\pagestyle{fancy}
	\fancyhead[LO,RE]{C-\thepage}
	\fancyhead[RO,LE]{Anhang C: Programm Plattenelemente}
	\addcontentsline{toc}{part}{Anhang C: Programm Plattenelemente}
	\section*{Anhang C: Programm Plattenelemente}
	\subsection*{Hauptprogramm konforme Plattenelemente}
	\begin{lstlisting}
	clear
	clc
	close
	%###########################################
	% main Program
	%###########################################
	% parameter
	E=2.1e11;       % N/m^2
	A=0.0001;       % m^2
	L=2;            % m  length of plate
	B=2;            % m  wide of plate
	rho=7850;       % Dichte in [kg/m^3]
	mu=rho*A;       % Massenbelegung in [kg/m]
	Nelx=15;        % number of elements per rand
	Nely=15;
	Nnox=Nelx+1;    % number of nodes per rand
	Nnoy=Nely+1;
	Nel_all=Nelx * Nely;  % total elements
	Nno_all=Nnox * Nnoy;  % total nodes
	lex=L/Nelx;           % length of an element
	ley=B/Nely;
	H=0.01;         % m Dicke von Platte
	v=0.30;         % Poissionzahl
	
	% define empty matrice
	Kt=zeros(Nno_all*4);     % empty global stiffnes-matrix 
	M=zeros(Nno_all*4);      % empty global mass-matrix 
	
	% call element routinesmbclient
	[Kte,Me] = Elementroutine_Platten(H,E,rho,lex,ley,v);
	% loop over every element
	r=0;
	a=zeros(4,1);
	b=a;
	c=[1 5 9 13];
	d=[4 8 12 16];
	for j=1 : Nely                                  
	 for k=1:Nelx
	  a(1)=((j-1)*Nnox+k-1)*4+1;
	  b(1)=a(1)+3;
	  a(2)=((j-1)*Nnox+k)*4+1;
	  b(2)=a(2)+3;
	  a(3)=((j*Nnox)+k-1)*4+1;
	  b(3)=a(3)+3;
	  a(4)=((j*Nnox)+k)*4+1;
	  b(4)=a(4)+3;
	  for m=1:4
	   for n=1:4
	    M(a(m):b(m), a(n):b(n)) = M(a(m):b(m), a(n):b(n)) + Me(c(m):d(m),c(n):d(n));
	    Kt(a(m):b(m), a(n):b(n)) = Kt(a(m):b(m), a(n):b(n)) + Kte(c(m):d(m),c(n):d(n));
	   end
	  end
	 end
	end
	
	% implementation of essetial boundary conditions
	M(1:Nnox*4,:) = [];
	M(:,1:Nnox*4) = [];
	
	Kt(1:Nnox*4,:) = [];
	Kt(:,1:Nnox*4) = [];
	
	% define system-matrix
	null=zeros(size(M));
	Eins = eye(size(M));
	SysMat=[null,Eins; -inv(M)*Kt,null];
	
	% compute Eigenvalues
	[V,D]=eig(SysMat);
	
	temp_d = diag(D);
	[nd, sortindex] = sort(temp_d);
	temp_v = V(:,sortindex);
	temp_f = imag(nd)/(2*pi);
	
	%% Mode Plot
	% vector with node coordinates
	xVec = zeros (Nnox, 1);     
	yVec = zeros (Nnoy, 1); 
	for k = 1: Nnox
	 xVec(k)=L/Nelx*(k-1);    
	end
	
	for g = 1: Nnoy
	 yVec(g)=B/Nely*(g-1);    
	end
	
	% delete lover part von eigenvectors 
	EigMat=temp_v(1:length(M),:) ;   
	EigMat=imag(EigMat);
	% delete unnecessary rows 
	for k = 1: length(M)/4      
	 EigMat(1+k:1+k+2,:)=[];       
	end
	% delete unnecessary colums
	for k = 1: length(M)
	 EigMat(:,k)=[];         
	end
	% fill up EigMat
	p = zeros(Nnox,Nnoy-1,9);
	for i = 1:9
	 for j = 1:Nnoy-1
	  for k = 1:Nnox
	   p(k,j,i) = EigMat(k+(j-1)*Nnox,i);
	  end
	 end
	end
	
	for i = 1 : 9
	 p1(:,:,i) = [zeros(Nnox,1),p(:,:,i)];
	end
	
	figure
	hold on
	for k = 1:9
	 subplot(3,3,k);
	 surf(yVec,xVec,p1(:,:,k));
	 set(gca,'FontSize',20);
	 title([num2str(k),'-Mode']);
	end
	\end{lstlisting}
	
	\subsubsection*{Elementrechnung konforme Plattenelemente}
	\begin{lstlisting}
	%###########################################
	% Elementroutine
	%###########################################
	function [Kte,Me] = Elementroutine_Platten(H,E,rho,lex,ley,v)
	% Elementroutine: compute Kte, Me
	% define empty Kte
	Kte=zeros(16);
	
	% define empty Me
	Me=zeros(16);
	
	% define sampling points for Gauss-quadrature
	xiVec=[-sqrt(3/7+2/7*sqrt(6/5)),-sqrt(3/7-2/7*sqrt(6/5)),sqrt(3/7-2/7*sqrt(6/5)),sqrt(3/7+2/7*sqrt(6/5))];
	etaVec=[-sqrt(3/7+2/7*sqrt(6/5)),-sqrt(3/7-2/7*sqrt(6/5)),sqrt(3/7-2/7*sqrt(6/5)),sqrt(3/7+2/7*sqrt(6/5))];
	% weights for sampling points of Gauss-quadrature
	wVec=[(18-sqrt(30))/36,(18+sqrt(30))/36,(18+sqrt(30))/36,(18-sqrt(30))/36];
	uVec=wVec;
	
	for i=1:length(xiVec)
	 x = xiVec(i);
	 w = wVec(i);
	 for j = 1:length(etaVec)
	  y = etaVec(j);
	  u = uVec(j);        
	  % define N, Nxx, Nyy, Nxy vector
	N=[ (1/16)*(-1 + x)^2*(2 + x)*(-1 + y)^2*(2 + y)  ...
	(1/16)*(-1 + x)^2*(1 + x)*(-1 + y)^2*(2 + y)  ...
	(1/16)*(-1 + x)^2*(2 + x)*(-1 + y)^2*(1 + y)  ...
	(1/16)*(-1 + x)^2*(1 + x)*(-1 + y)^2*(1 + y) ...
	(-(1/16))*(-2 + x)*(1 + x)^2*(-1 + y)^2*(2 + y) ...
	(1/16)*(-1 + x)*(1 + x)^2*(-1 + y)^2*(2 + y) ...
	(-(1/16))*(-2 + x)*(1 + x)^2*(-1 + y)^2*(1 + y) ...
	(1/16)*(-1 + x)*(1 + x)^2*(-1 + y)^2*(1 + y) ...
	(-(1/16))*(-1 + x)^2*(2 + x)*(-2 + y)*(1 + y)^2 ...
	(-(1/16))*(-1 + x)^2*(1 + x)*(-2 + y)*(1 + y)^2 ...
	(1/16)*(-1 + x)^2*(2 + x)*(-1 + y)*(1 + y)^2 ...
	(1/16)*(-1 + x)^2*(1 + x)*(-1 + y)*(1 + y)^2 ...
	(1/16)*(-2 + x)*(1 + x)^2*(-2 + y)*(1 + y)^2 ...
	(-(1/16))*(-1 + x)*(1 + x)^2*(-2 + y)*(1 + y)^2 ...
	(-(1/16))*(-2 + x)*(1 + x)^2*(-1 + y)*(1 + y)^2 ...
	(1/16)*(-1 + x)*(1 + x)^2*(-1 + y)*(1 + y)^2 ];
	
	Nxx=[ (3*x*(-1 + y)^2*(2 + y))/(2*lex^2) ...
	((-1 + 3*x)*(-1 + y)^2*(2 + y))/(2*lex^2) ...
	(3*x*(-1 + y)^2*(1 + y))/(2*lex^2) ...
	((-1 + 3*x)*(-1 + y)^2*(1 + y))/(2*lex^2) ...
	-((3*x*(-1 + y)^2*(2 + y))/(2*lex^2)) ...
	((1 + 3*x)*(-1 + y)^2*(2 + y))/(2*lex^2) ...
	-((3*x*(-1 + y)^2*(1 + y))/(2*lex^2)) ...
	((1 + 3*x)*(-1 + y)^2*(1 + y))/(2*lex^2) ...
	-((3*x*(-2 + y)*(1 + y)^2)/(2*lex^2)) ...
	-(((-1 + 3*x)*(-2 + y)*(1 + y)^2)/(2*lex^2)) ...
	(3*x*(-1 + y)*(1 + y)^2)/(2*lex^2) ...
	((-1 + 3*x)*(-1 + y)*(1 + y)^2)/(2*lex^2) ...
	(3*x*(-2 + y)*(1 + y)^2)/(2*lex^2) ...
	-(((1 + 3*x)*(-2 + y)*(1 + y)^2)/(2*lex^2)) ...
	-((3*x*(-1 + y)*(1 + y)^2)/(2*lex^2)) ...
	((1 + 3*x)*(-1 + y)*(1 + y)^2)/(2*lex^2) ];
	
	Nyy=[ (3*(-1 + x)^2*(2 + x)*y)/(2*ley^2) ...
	(3*(-1 + x)^2*(1 + x)*y)/(2*ley^2) ...
	((-1 + x)^2*(2 + x)*(-1 + 3*y))/(2*ley^2) ...
	((-1 + x)^2*(1 + x)*(-1 + 3*y))/(2*ley^2) ...
	-((3*(-2 + x)*(1 + x)^2*y)/(2*ley^2)) ...
	(3*(-1 + x)*(1 + x)^2*y)/(2*ley^2) ...
	-(((-2 + x)*(1 + x)^2*(-1 + 3*y))/(2*ley^2)) ...
	((-1 + x)*(1 + x)^2*(-1 + 3*y))/(2*ley^2) ...
	-((3*(-1 + x)^2*(2 + x)*y)/(2*ley^2)) ...
	-((3*(-1 + x)^2*(1 + x)*y)/(2*ley^2)) ...
	((-1 + x)^2*(2 + x)*(1 + 3*y))/(2*ley^2) ...
	((-1 + x)^2*(1 + x)*(1 + 3*y))/(2*ley^2) ...
	(3*(-2 + x)*(1 + x)^2*y)/(2*ley^2) ...
	-((3*(-1 + x)*(1 + x)^2*y)/(2*ley^2)) ...
	-(((-2 + x)*(1 + x)^2*(1 + 3*y))/(2*ley^2)) ...
	((-1 + x)*(1 + x)^2*(1 + 3*y))/(2*ley^2) ];
	
	Nxy=[ (9*(-1 + x^2)*(-1 + y^2))/(4*lex*ley) ...
	(3*(-1 + x)*(1 + 3*x)*(-1 + y^2))/(4*lex*ley) ...
	(3*(-1 + x^2)*(-1 + y)*(1 + 3*y))/(4*lex*ley) ...
	((-1 + x)*(1 + 3*x)*(-1 + y)*(1 + 3*y))/(4*lex*ley) ...
	-((9*(-1 + x^2)*(-1 + y^2))/(4*lex*ley)) ...
	(3*(1 + x)*(-1 + 3*x)*(-1 + y^2))/(4*lex*ley) ...
	-((3*(-1 + x^2)*(-1 + y)*(1 + 3*y))/(4*lex*ley)) ...
	((1 + x)*(-1 + 3*x)*(-1 + y)*(1 + 3*y))/(4*lex*ley) ...
	-((9*(-1 + x^2)*(-1 + y^2))/(4*lex*ley)) ...
	-((3*(-1 + x)*(1 + 3*x)*(-1 + y^2))/(4*lex*ley)) ...
	(3*(-1 + x^2)*(1 + y)*(-1 + 3*y))/(4*lex*ley) ...
	((-1 + x)*(1 + 3*x)*(1 + y)*(-1 + 3*y))/(4*lex*ley) ...
	(9*(-1 + x^2)*(-1 + y^2))/(4*lex*ley) ...
	-((3*(1 + x)*(-1 + 3*x)*(-1 + y^2))/(4*lex*ley)) ...
	-((3*(-1 + x^2)*(1 + y)*(-1 + 3*y))/(4*lex*ley)) ...
	((1 + x)*(-1 + 3*x)*(1 + y)*(-1 + 3*y))/(4*lex*ley) ];
	
	 % compute Kte and Me for sampling point of Gauss-integration
	 Me=Me + rho * H * (N' * N) * w * u;
	 Kte=Kte + ((E*H^3)/(12*(1-v^2))) * ((Nxx' * Nxx)+v*((Nxx'*Nyy)+(Nyy'*Nxx))+(Nyy'*Nyy)+2*(1-v)*(Nxy'*Nxy)) * w * u;
	 end
	 end
	
	end
	\end{lstlisting}